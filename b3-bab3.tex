%==================================================================
% Ini adalah bab 3
% Silahkan edit sesuai kebutuhan, baik menambah atau mengurangi \section, \subsection
%==================================================================

\chapter[PELAKSANAAN DAN PEMBAHASAN PRAKTIK INDUSTRI]{\\ PELAKSANAAN DAN PEMBAHASAN PRAKTIK INDUSTRI}

\section{Profil Divisi/Departemen Tempat Praktik}
Pada bagian ini, mahasiswa mendeskripsikan struktur organisasi divisi tempat praktik industri dilaksanakan. Uraian dimulai dengan gambaran umum tentang posisi divisi dalam keseluruhan struktur perusahaan, menjelaskan hierarki dan hubungan antarunit kerja. Mahasiswa perlu menggambarkan secara detail fungsi dan peran strategis divisi, fokus pada aspek teknis seperti Research and Development (R\&D), Produksi, Maintenance, atau Engineering.

Deskripsi harus mencakup pembagian tugas, tanggung jawab setiap unit, dan kontribusi divisi terhadap pencapaian tujuan perusahaan. Jika memungkinkan, sertakan bagan struktur organisasi untuk memperjelas penjelasan. Mahasiswa diharapkan mampu menunjukkan pemahaman mendalam tentang mekanisme kerja dan kompleksitas organisasi di lingkungan industri.

\section{Ruang Lingkup Kegiatan Praktik}
Bagian ini menjelaskan secara komprehensif ruang lingkup kegiatan praktik yang dilakukan. Mahasiswa harus mendefinisikan dengan spesifik bidang fokus kegiatan, apakah berkaitan dengan Elektronikanika industri, sistem kontrol, instrumentasi, atau bidang khusus lainnya dalam lingkup Teknik Elektronika.

Uraian mencakup penjelasan detail tentang lingkup pekerjaan, teknologi yang digunakan, dan sistem yang menjadi fokus selama praktik. Mahasiswa perlu menunjukkan relevansi ruang lingkup kegiatan dengan kompetensi keilmuan Teknik Elektronika, serta menggambarkan batasan dan karakteristik pekerjaan yang dilakukan.

\section{Deskripsi Kegiatan Praktik}
Pada bagian ini, mahasiswa menguraikan jenis kegiatan praktik secara kronologis dan sistematis. Deskripsi harus disajikan dengan bahasa ilmiah dan teknis, menjelaskan metode pelaksanaan pekerjaan secara detail. Mahasiswa diharapkan mampu menggambarkan alur kerja, prosedur yang diikuti, dan kontribusi personal dalam setiap kegiatan. Penjelasan Alat dan Bahan juga harus dicantumkan pada bagian ini secara lengkap seperti spesifikasinya maupun foto dari alat dan bahan yang digunakan. Penjelasan terkait kesehatan dan keselamatan kerja juga harus dicantumkan pada bagian ini termasuk APD yang digunakan.

Penjelasan dapat dilengkapi dengan diagram atau flowchart untuk memperjelas alur kerja. Fokus utama adalah menunjukkan proses kerja, standar yang diterapkan, dan kompleksitas tugas yang dilakukan selama praktik industri. Selain itu bagian ini juga harus dilengkapi dengan jadwal kegiatan lengkap.

\section{Dokumentasi Kegiatan}
Bagian dokumentasi bertujuan menghadirkan bukti visual dan dokumenter kegiatan praktik. Mahasiswa perlu memilih dan menyertakan foto-foto kegiatan yang relevan dan bermakna, dengan fokus pada proses teknis dan aktivitas nyata di lapangan. Setiap foto atau dokumen harus disertai keterangan yang jelas.

Dokumentasi yang dilampirkan dapat mencakup surat penugasan, laporan harian/mingguan, dokumentasi hasil pekerjaan, dan bukti-bukti lain yang menguatkan laporan. Pastikan kualitas dokumentasi baik dan memiliki nilai informatif.

\subsection{Dokumentas Kegiatan Pertama}
Penjelasan spesifik untuk kegiatan atau project.

\subsection{Dokumentas Kegiatan Kedua}
Penjelasan spesifik untuk kegiatan atau project dan dapat ditambahkan sesuai kebutuhan.

\section{Analisis Teknis Kegiatan}
Pada bagian analisis teknis, mahasiswa mengidentifikasi permasalahan teknis yang dihadapi selama praktik. Pendekatan ilmiah menjadi kunci dalam menganalisis persoalan, dengan menjelaskan metode penyelesaian masalah secara sistematis. Mahasiswa perlu menunjukkan perhitungan, pertimbangan teknis, dan solusi yang diterapkan.

Analisis harus mencerminkan kemampuan berpikir kritis dan problem solving, dengan menggunakan referensi atau teori pendukung. Fokus pada inovasi atau kontribusi yang diberikan dalam menyelesaikan permasalahan teknis.

\subsection{Kegiatan Pertama}
Penjelasan spesifik untuk kegiatan atau project.

\subsection{Kegiatan Kedua}
Penjelasan spesifik untuk kegiatan atau project dan dapat ditambahkan sesuai kebutuhan.

\section{Kompetensi dan Keterampilan yang Dikembangkan}
Bagian ini mengkategorikan kompetensi menjadi hard skills (keterampilan teknis) dan soft skills (keterampilan non-teknis). Mahasiswa menjelaskan secara detail pengembangan keterampilan selama praktik, memberikan contoh konkret untuk setiap kategori.

Uraian perlu menghubungkan kompetensi yang dikembangkan dengan capaian pembelajaran program studi. Refleksi tentang tantangan dan pembelajaran menjadi bagian penting dalam menggambarkan pertumbuhan profesional.

\section{Pembahasan dan Evaluasi}
Pada bagian pembahasan, mahasiswa membandingkan teori yang dipelajari dengan praktik nyata. Evaluasi dilakukan secara kritis dan objektif, dengan mengidentifikasi kesenjangan antara teori dan implementasi di lapangan. 

Mahasiswa menjelaskan tantangan yang dihadapi, solusi yang diterapkan, serta menggunakan referensi atau standar industri sebagai acuan analisis. Refleksi harus menunjukkan kemampuan berpikir analitis dan kemampuan adaptasi.

\section{Kontribusi praktik Industri terhadap Pengembangan Kompetensi}
Bagian terakhir menunjukkan keterkaitan praktik dengan kurikulum program studi. Mahasiswa menjelaskan manfaat praktik industri untuk pengembangan karier, memberikan rekomendasi perbaikan kurikulum berdasarkan pengalaman di lapangan.

Refleksi difokuskan pada nilai tambah praktik industri, proyeksi pengembangan kompetensi di masa depan, serta kontribusi pengalaman praktik dalam membentuk profesionalitas mahasiswa Teknik Elektronika.

Pada bagian ini juga menjelaskan peluang jenis mata kuliah yang dapat dijadikan konversi kegiatan \tipe ini secara lengkap dengan bukti-bukti pendukung yang kuat.