%==================================================================
% Ini adalah bab 2
% Silahkan edit sesuai kebutuhan, baik menambah atau mengurangi \section, \subsection
%==================================================================

\chapter[PROFIL PERUSAHAAN]{\\ PROFIL PERUSAHAAN}

\section{Sejarah \perusahaan}
Sejarah perusahaan adalah narasi yang mencerminkan perjalanan, perkembangan, dan pencapaian suatu entitas bisnis dari awal pendiriannya hingga saat ini. [Nama Perusahaan] didirikan pada [tahun pendirian] oleh [nama pendiri atau pendiri-pendiri]. Dengan visi dan misi yang jelas, perusahaan ini telah melalui berbagai tahap perkembangan yang memainkan peran krusial dalam membentuk identitasnya.

Pada awalnya, [Nama Perusahaan] mungkin dimulai sebagai usaha kecil dengan fokus pada [jenis produk atau layanan]. Seiring waktu, dedikasi dan komitmen terhadap kualitas mendorong perusahaan untuk berkembang pesat. Pada [tahun-tahun tertentu], [Nama Perusahaan] mungkin mengalami ekspansi pasar atau diversifikasi portofolio produk dan layanannya.

Perjalanan perusahaan mungkin juga mencakup tantangan dan perubahan strategis. Keterlibatan dalam inovasi, adaptasi terhadap perkembangan teknologi, dan respons terhadap perubahan pasar merupakan elemen-elemen penting dalam sejarah perusahaan ini. [Nama Perusahaan] dapat memiliki sejumlah pencapaian dan tonggak sejarah, seperti peluncuran produk inovatif, penerimaan penghargaan industri, atau ekspansi internasional yang mencirikan ketekunan dan ketangguhannya.

Sebagai bagian integral dari komunitas bisnis, [Nama Perusahaan] mungkin juga terlibat dalam berbagai inisiatif sosial atau keberlanjutan yang mencerminkan tanggung jawab korporatifnya. Sejarah perusahaan ini mencerminkan komitmen terhadap nilai-nilai inti dan tujuan jangka panjang untuk pertumbuhan berkelanjutan.

Dengan melihat sejarah perusahaan, seseorang dapat memahami fondasi, evolusi, dan prinsip-prinsip yang membimbing [Nama Perusahaan] hingga menjadi kekuatan yang diakui dalam industri ini. Sejarah ini tidak hanya menceritakan perjalanan sukses, tetapi juga merupakan cerminan dari nilai-nilai yang dianut oleh [Nama Perusahaan] dalam memenuhi kebutuhan pelanggan dan memberikan dampak positif bagi masyarakat.

Contoh bullet item:
\begin{packed_item}
    \item Teori-teori yang digunakan dalam penelitian
    \item Konsep-konsep yang digunakan dalam penelitian
    \item Kerangka teori yang digunakan dalam proyek akhir sarjana terapan
\end{packed_item}

Contoh penomoran:
\begin{packed_enum}
	\item Teori-teori yang digunakan dalam penelitian
	\item Konsep-konsep yang digunakan dalam penelitian
	\item Kerangka teori yang digunakan dalam proyek akhir sarjana terapan
\end{packed_enum}

\subsection{Sub Dasar Teori 2.1.1}
Bagian ini digunakan apabila dibutuhkan, silahkan bisa ditambah atau dikurangi sesuai kebutuhan.

\subsection{Sub Dasar Teori 2.1.2}
Bagian ini digunakan apabila dibutuhkan, silahkan bisa ditambah atau dikurangi sesuai kebutuhan.

\subsection{Sub Dasar Teori 2.1.3}
Bagian ini digunakan apabila dibutuhkan, silahkan bisa ditambah atau dikurangi sesuai kebutuhan.

\section{Dasar Teori 2.2}
\lipsum

\subsection{Sub Dasar Teori 2.2.1}
Bagian ini digunakan apabila dibutuhkan, silahkan bisa ditambah atau dikurangi sesuai kebutuhan.

\subsection{Sub Dasar Teori 2.2.2}
Bagian ini digunakan apabila dibutuhkan, silahkan bisa ditambah atau dikurangi sesuai kebutuhan.

\subsection{Sub Dasar Teori 2.2.3}
Bagian ini digunakan apabila dibutuhkan, silahkan bisa ditambah atau dikurangi sesuai kebutuhan.